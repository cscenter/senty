\documentclass[a4paper, 12pt]{article}

\usepackage{cmap} 
\usepackage[T2A]{fontenc} 
\usepackage[utf8]{inputenc} 
\usepackage[english, russian]{babel}

%%% Математика 
\usepackage{amssymb} \usepackage{dsfont}

%%% Поля документа 
\usepackage{geometry} 
\geometry{top = 10mm} 
\geometry{bottom = 20mm} 
\geometry{left = 20mm}
\geometry{right = 20mm}

%%% Красная строка 
\usepackage{indentfirst}

%%% Заголовок 
\author{Сухочев Александр и Балакший Андрей} 
\title{Результаты наблюдений} 

%%% Теоремы 
\usepackage{amsthm} 
\theoremstyle{plain} 
\newtheorem{proposition}{Утверждение}[section]

\theoremstyle{definition} 
\newtheorem{theorem}{Теорема}[proposition]

\theoremstyle{remark} 
\newtheorem{remarks}{Следствие}[section]

%%% Начало документа
\begin{document}
\maketitle

Гляньте на эту табличку!

Logistic Regression with 1 0

Accuracy: 0.663265306122

All test count: 98; TP: 7; TN: 58; FP: 10; FN: 23

и

Logistic Regression with tf idf

Accuracy: 0.704081632653

All test count: 98; TP: 1; TN: 68; FP: 0; FN: 29

С Тф идф выдает лучший результат, но посмотрите на ТП! Просто так совпало что у нас больше негативных в тестирующем множестве, поэтому мы не можем сказать что тф идф лучше работает, видно ведь что он хуже определяет Позитивные! Переделывать обучающее множество нельзя, т.е. у нас и при работе программы будет подобное соотношение позитивных/негативных, но спасибо за полезный опыт

\end{document}